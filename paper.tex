\documentclass{chi-ext}
% Please be sure that you have the dependencies (i.e., additional LaTeX packages) to compile this example.
% See http://personales.upv.es/luileito/chiext/

%% EXAMPLE BEGIN -- HOW TO OVERRIDE THE DEFAULT COPYRIGHT STRIP -- (July 22, 2013 - Paul Baumann)
% \copyrightinfo{Permission to make digital or hard copies of all or part of this work for personal or classroom use is granted without fee provided that copies are not made or distributed for profit or commercial advantage and that copies bear this notice and the full citation on the first page. Copyrights for components of this work owned by others than ACM must be honored. Abstracting with credit is permitted. To copy otherwise, or republish, to post on servers or to redistribute to lists, requires prior specific permission and/or a fee. Request permissions from permissions@acm.org. \\
% {\emph{IHM14}}, 28-31 Octobre, 2014, Lille, France. \\
% Copyright \copyright~2014 ACM ISBN/14/04...\$15.00. \\
% DOI string from ACM form confirmation}
%% EXAMPLE END -- HOW TO OVERRIDE THE DEFAULT COPYRIGHT STRIP -- (July 22, 2013 - Paul Baumann)

\title{Test of Adaptable User Interfaces}

\numberofauthors{4}
% Notice how author names are alternately typesetted to appear ordered in 2-column format;
% i.e., the first 4 autors on the first column and the other 4 auhors on the second column.
% Actually, it's up to you to strictly adhere to this author notation.
\author{
  \alignauthor{
  	\textbf{Nelson}\\
  	\affaddr{AuthorCo, Inc.}\\
  	\affaddr{123 Author Ave.}\\
  	\affaddr{Authortown, PA 54321 USA}\\
  	\email{author1@anotherco.com}
  }\alignauthor{
  	\textbf{Julien}\\
  	\affaddr{AuthorCo, Inc.}\\
  	\affaddr{123 Author Ave.}\\
  	\affaddr{Authortown, PA 54321 USA}\\
  	\email{author5@anotherco.com}
  }
  \vfil
  \alignauthor{
  	\textbf{Lydie}\\
  	\affaddr{AuthorCo, Inc.}\\
  	\affaddr{123 Author Ave.}\\
  	\affaddr{Authortown, PA 54321 USA}\\
  	\email{author2@anotherco.com}
  }\alignauthor{
  	\textbf{Sophie}\\
  	\affaddr{AuthorCo, Inc.}\\
  	\affaddr{123 Author Ave.}\\
  	\affaddr{Authortown, PA 54321 USA}\\
  	\email{author6@anotherco.com}
  }
}

% Paper metadata (use plain text, for PDF inclusion and later re-using, if desired)
\def\plaintitle{Test of Adaptable User Interfaces}
\def\plainauthor{Nelson Mariano Leite Neto}
\def\plainkeywords{Guides, instructions, author's kit, conference publications}
\def\plaingeneralterms{Documentation, Standardization}

\hypersetup{
  % Your metadata go here
  pdftitle={\plaintitle},
  pdfauthor={\plainauthor},  
  pdfkeywords={\plainkeywords},
  pdfsubject={\plaingeneralterms},
  % Quick access to color overriding:
  %citecolor=black,
  %linkcolor=black,
  %menucolor=black,
  %urlcolor=black,
}

\usepackage{graphicx}   % for EPS use the graphics package instead
\usepackage{balance}    % useful for balancing the last columns
\usepackage{bibspacing} % save vertical space in references
%\usepackage{hyperref}
%\hypersetup{colorlinks=false}

%\widowpenalty=10000
%\clubpenalty=10000

\begin{document}

\maketitle

\begin{abstract}
TODO - Lorem ipsum sed ut perspiciatis unde omnis iste natus error sit voluptatem accusantium doloremque laudantium, totam rem aperiam, eaque ipsa quae ab illo inventore veritatis et quasi architecto beatae vitae dicta sunt explicabo. Nemo enim ipsam voluptatem quia voluptas sit aspernatur aut odit aut fugit, sed quia consequuntur magni dolores eos qui ratione voluptatem sequi nesciunt. Neque porro quisquam est, qui dolorem ipsum quia dolor sit amet, consectetur, adipisci velit, sed quia non numquam eius modi tempora incidunt ut labore et dolore magnam aliquam quaerat voluptatem.
\end{abstract}

\keywords{\plainkeywords}
\textcolor{red}{Champs requis dans la version finale}

\category{H.5.m}{Information interfaces and presentation (e.g., HCI)}{Miscellaneous}. 
%See \cite{ACMCCS} 
Voir : \url{http://www.acm.org/about/class/1998/} 
pour une description du ACM Classification system.
\textcolor{red}{Champs requis dans la version finale}

\terms{\plaingeneralterms}
\textcolor{red}{Champs requis dans la version finale}


% =============================================================================
\section{Introduction}
% =============================================================================
Nowadays, with the rise of mobile devices such as notebooks, smartphones, smartwatches and tablets, there is computer systems running everywhere and in different \textit{contexts of use}, which involves three factors: the \textit{platform}, \textit{environment} and \textit{the user}~\cite{ui-eric-plasticity-vs-responsive}. These factors affect the interaction between the user and the system. This interaction happens through the \textit{user interface} (UI), and users have to deal with different UIs adapted to several contexts. Hence the term \textit{adaptable user interfaces}. All this allows to see that today developing UIs can be as complex as developing the core of the system. There are several questions to be considered, among them: quality. \textit{Software Testing} is one of the most used ways in the industry to assure quality~\cite{test-hdr-lydie}. It can be defined as a process, or a series of processes, designed to ensure computer code does what it was designed to do and, conversely, that it does not anything unintended~\cite{test-art-of-testing}. What is the impact of adaptable user interfaces on software testing? Are there already tools to test these adaptable UIs?

Answering these questions motivates the development of this project. In this work, we focus on the scenario of having different UIs for the same system. These UIs can be adapted to different devices or have different versions for the same device. For instance, we used as case study a smart home energy management system prototype with four equivalent versions: three for web and one for mobile environment. This exposes the problem which we focus: the repetitive work of writing and maintaining test scripts for different versions of the UIs of the same system.

Our contribution is the approach of automatically generate test scripts for each version of the UI of the same system from scripts written in a higher-level language. In other words, testers would be able to write test cases for all the versions using abstract descriptions. A tool would be responsible for interpreting these descriptions and generating test scripts ready to be executed for each version of the UI.

This paper is structured as follows. First, the related work is presented, explaining several concepts about test automation, UI testing and tools. Then, we explain the case study used in this research. Next, we show how test scripts can be generated from abstract descriptions using an existing tool. Finally, the conclusions that summarize our current results and proposed perspectives.

% =============================================================================
\section{Related Work}
% =============================================================================
With the popularity of Continuous Integration (CI), a software engineering practice in which isolated changes are immediately tested and reported on when they are added to a larger code base, the need of test automation has increased. Test automation can be achieved by three ways: the generation of test scripts, their execution and the analysis of the results. The scientific community has developed several approaches and tools for automation of regular and UI testing. Some of them will be presented below.

An example of automation for the generation of test scripts is Tobias~\cite{tobias}, a combinatorial tool that instantiates a large set of test sequences from an abstract description. Combinatorial testing allows to express a large set of test cases in a few lines. Tobias receives the abstract test and generates an executable test code containing several possible combinations of execution, according to the specification of the abstract code given. 

In the case of execution automation, several tools have been developed focusing on UI testing. There are tools that drive the UI as users do, triggering actions, navigating the system and checking the UI content. For instance, tools for web~\cite{watir, cucumber} and mobile~\cite{appium, selendroid} environments are responsible for interacting with the UI according to the scripts written by the testers. Besides, there are tools~\cite{selenium, monkeytalk} that are also able to generate test scripts, through the process of record and playback. The user interacts with the system, the tool records the actions, thus generating a test script that can be used thereafter for execution automation.

Another interesting approach, proposed in~\cite{test-image-comparison-korea}, uses the idea of record and playblack with the automation of the analysis of the results through image comparison. In this solution, after the execution of a test script, the image of the UI is recorded. It can be used later to verify if changes in the system affected the UI, comparing automatically the old and new images of the UI.
 
Another related work, presented in~\cite{ui-equilavance-raquel}, is not strictly about testing, but comparing UIs of the same system that are adapted to different contexts. In this approach, the UIs are abstracted to formal models based on graphs and the abstractions can be automatically compared, verifying if  they have the same capabilities and appearance.

% =============================================================================
\section{Case Study}
% =============================================================================
TODO

% =============================================================================
\section{Approach: generating test scripts for UI testing from abstract descriptions}
% =========================================
\subsection{Principle}
TODO

\subsection{Principles put into Practice}
TODO

\subsection{Discussion and Analysis}
TODO

% =============================================================================
\section{Conclusions and Perspectives}
% =============================================================================
TODO

\balance
\bibliographystyle{acm-sigchi}
\bibliography{ref}

\end{document}