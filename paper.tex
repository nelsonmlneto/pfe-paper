\documentclass{chi-ext}
% Please be sure that you have the dependencies (i.e., additional LaTeX packages) to compile this example.
% See http://personales.upv.es/luileito/chiext/

%% EXAMPLE BEGIN -- HOW TO OVERRIDE THE DEFAULT COPYRIGHT STRIP -- (July 22, 2013 - Paul Baumann)
% \copyrightinfo{Permission to make digital or hard copies of all or part of this work for personal or classroom use is granted without fee provided that copies are not made or distributed for profit or commercial advantage and that copies bear this notice and the full citation on the first page. Copyrights for components of this work owned by others than ACM must be honored. Abstracting with credit is permitted. To copy otherwise, or republish, to post on servers or to redistribute to lists, requires prior specific permission and/or a fee. Request permissions from permissions@acm.org. \\
% {\emph{IHM14}}, 28-31 Octobre, 2014, Lille, France. \\
% Copyright \copyright~2014 ACM ISBN/14/04...\$15.00. \\
% DOI string from ACM form confirmation}
%% EXAMPLE END -- HOW TO OVERRIDE THE DEFAULT COPYRIGHT STRIP -- (July 22, 2013 - Paul Baumann)

\title{Test of Adaptable User Interfaces}

\numberofauthors{4}
% Notice how author names are alternately typesetted to appear ordered in 2-column format;
% i.e., the first 4 autors on the first column and the other 4 auhors on the second column.
% Actually, it's up to you to strictly adhere to this author notation.
\author{
  \alignauthor{
  	\textbf{Nelson}\\
  	\affaddr{AuthorCo, Inc.}\\
  	\affaddr{123 Author Ave.}\\
  	\affaddr{Authortown, PA 54321 USA}\\
  	\email{author1@anotherco.com}
  }\alignauthor{
  	\textbf{Julien}\\
  	\affaddr{AuthorCo, Inc.}\\
  	\affaddr{123 Author Ave.}\\
  	\affaddr{Authortown, PA 54321 USA}\\
  	\email{author5@anotherco.com}
  }
  \vfil
  \alignauthor{
  	\textbf{Lydie}\\
  	\affaddr{AuthorCo, Inc.}\\
  	\affaddr{123 Author Ave.}\\
  	\affaddr{Authortown, PA 54321 USA}\\
  	\email{author2@anotherco.com}
  }\alignauthor{
  	\textbf{Sophie}\\
  	\affaddr{AuthorCo, Inc.}\\
  	\affaddr{123 Author Ave.}\\
  	\affaddr{Authortown, PA 54321 USA}\\
  	\email{author6@anotherco.com}
  }
}

% Paper metadata (use plain text, for PDF inclusion and later re-using, if desired)
\def\plaintitle{Test of Adaptable User Interfaces}
\def\plainauthor{Nelson Mariano Leite Neto}
\def\plainkeywords{Guides, instructions, author's kit, conference publications}
\def\plaingeneralterms{Documentation, Standardization}

\hypersetup{
  % Your metadata go here
  pdftitle={\plaintitle},
  pdfauthor={\plainauthor},  
  pdfkeywords={\plainkeywords},
  pdfsubject={\plaingeneralterms},
  % Quick access to color overriding:
  %citecolor=black,
  %linkcolor=black,
  %menucolor=black,
  %urlcolor=black,
}

\usepackage{graphicx}   % for EPS use the graphics package instead
\usepackage{balance}    % useful for balancing the last columns
\usepackage{bibspacing} % save vertical space in references
%\usepackage{hyperref}
%\hypersetup{colorlinks=false}

%\widowpenalty=10000
%\clubpenalty=10000

\begin{document}

\maketitle

\begin{abstract}
TODO - Lorem ipsum sed ut perspiciatis unde omnis iste natus error sit voluptatem accusantium doloremque laudantium, totam rem aperiam, eaque ipsa quae ab illo inventore veritatis et quasi architecto beatae vitae dicta sunt explicabo. Nemo enim ipsam voluptatem quia voluptas sit aspernatur aut odit aut fugit, sed quia consequuntur magni dolores eos qui ratione voluptatem sequi nesciunt. Neque porro quisquam est, qui dolorem ipsum quia dolor sit amet, consectetur, adipisci velit, sed quia non numquam eius modi tempora incidunt ut labore et dolore magnam aliquam quaerat voluptatem.
\end{abstract}

\keywords{\plainkeywords}
\textcolor{red}{Champs requis dans la version finale}

\category{H.5.m}{Information interfaces and presentation (e.g., HCI)}{Miscellaneous}. 
%See \cite{ACMCCS} 
Voir : \url{http://www.acm.org/about/class/1998/} 
pour une description du ACM Classification system.
\textcolor{red}{Champs requis dans la version finale}

\terms{\plaingeneralterms}
\textcolor{red}{Champs requis dans la version finale}


% =============================================================================
\section{Introduction}
% =============================================================================
TODO

% =============================================================================
\section{Related Work}
% =============================================================================
TODO

% =============================================================================
\section{Case Study}
% =============================================================================
TODO

% =============================================================================
\section{Approach: generating test scripts for UI testing from abstract descriptions}
% =========================================
\subsection{Principle}
TODO

\subsection{Principles put into Practice}
TODO

\subsection{Discussion and Analysis}
Firstly, the number of lines to write for tests have been at least halved, as Table \ref{table:scriptssize} shows it.
\begin{table}
\begin{tabular}{l l r r r r}
			&					&	mobile	&	web0	&	web1	&	web2	\\
goal		&	plain Java	&	161		&	143		&	151		&	147		\\
			&	with Tobias	&	76		&	67		&	69		&	68		\\
filter		&	plain Java	&	/		&	350		&	402		&	369		\\
			&	with Tobias	&	/		&	88		&	93		&	92		\\
compare	&	plain Java	&	/		&	/		&	235		&	227		\\
			&	with Tobias	&	/		&	/		&	59		&	58
\end{tabular}
\caption{Comparison of number of lines for various functionalities and various versions for Tobias script and its generated test}
\label{table:scriptssize}
\end{table}
Secondly, the logical structure of the various tests for a same functionality and version is shared, and so the differences between these files are very localized : in the header of the file (preferences for the test) and in the implementations of the abstract groups (version details). So there is little variation from a shared abstract script for each file, as can be seen in Table \ref{table:scripts-similarity}.
\begin{table}
\begin{tabular}{l l r r r}
			&			&	shared	&	total	&	\%	\\
goal		&	mobile	&	48		&	76		&	63	\\
			&	web0	&	65		&	67		&	97	\\
			&	web2	&	65		&	68		&	95	\\
filter		&	web0	&	78		&	88		&	88	\\
			&	web2	&	89		&	92		&	96	\\
compare	&	web2	&	55		&	58		&	94
\end{tabular}
\caption{Comparison of number of shared lines between Tobias scripts and version web1 for a same functionality}
\label{table:scripts-similarity}
\end{table}
When small changes are applied to the UI, the scripts only differ by some lines. Even in the case of a change of platform (web to mobile) the scripts stay similar.
Lastly, there is also shared lines between files, whichever their version and functionality tested, because these tests proceed by steps, which can sometimes be the same for several scripts.
So this approach allows to reduce the amount of work, while it facilitates it by reducing differences between script versions.

% =============================================================================
\section{Conclusions and Perspectives}
% =============================================================================
TODO

\balance
\bibliographystyle{acm-sigchi}
\bibliography{ref}

\end{document}